%%
%% Author: Louise
%% 08/05/2018
%%

% Preamble
\documentclass[11pt]{article}
\title{High Energy States in Superconducting Cavity Quantum Electrodynamics}
\author{Louise McCann \\ URN: 6307177 \\ Email: lm00293@surrey.ac.uk \\ Department of Physics, University of Surrey, Guildford, GU2 7XH, UK}


% Packages
\usepackage{a4wide}
\usepackage{physics}
\usepackage{graphicx}
\usepackage{amsmath}
\usepackage{amssymb}
\usepackage{caption}
\usepackage{subcaption}
\usepackage{epstopdf}

\newcommand*{\hatH}{\hat{\mathcal{H}}}
\graphicspath{C:/Users/Louise/Documents/project/final_project/figs/}

% Document
\begin{document}
\graphicspath{C:/Users/Louise/Documents/project/final_project/figs/}
\maketitle
\begin{center}
Supervisor: Eran Ginossar
\end{center}
\begin{abstract}

\end{abstract}

\clearpage

\section{Introduction}
The superconducting circuit has many uses in modern physics. Its applications include SQUIDS to detect subtle magnetic fields and qubits which are the underlying basis of quantum computers. Quantum computing and the methods behind it could have a big impact on the future of physics. A common circuit made of two superconducting reservoirs with a thin insulating layer between them is called a Josephson Junction. This was predicted by Brian Josephson who won a Nobel Prize for his work. A Josephson Junction works by allowing cooper pairs to tunnel across the barrier without any electrical resistance. Cooper pairs are bound electrons that form at very low temperatures and are the mechanism behind superconductivity. Josephson Junctions are relatively simple devices, which means they are easy to manufacture. They have long coherence times in principle, which means that many operations could be performed on a qubit made out of a Josephson Junction. The Junction is well known and has been studied in many applications. However, the majority of research has been focused on the lower states of the junction. This investigation aims to study the theory of the Josephson Junction and in particular to assess what the properties the system has when excited into higher states.

\subsection{Quantum Computers}
Quantum computers are based on the principle that a particle can be in a superposition of states. The theory that computers can be built using this principle has been around since the 1980s and yet the development of quantum computers is still ongoing. These computers would take advantage of quantum algorithms, such as. These could solve problems faster than modern computers and could solve extremely complex problems, such as the many-body problem. Quantum computers are built from qubits, or quantum mechanical circuits that can be measured to be either in the ground state or the first excited state. Qubits rely on the quantum mechanical principle, which states that the qubit is in both states simultaneously until measured. The furthest application has been achieved using NMR techniques to represent a qubit, but another popular medium is to use a superconducting circuit.

\subsection{Superconducting Circuits}
Superconductors are materials that, at low temperatures, have zero resistance to electronic fields. At these low temperatures, pairs of electrons become bound together in such a way that the energy is lower than that of a single electron. These are the charge carriers in a superconducting material. As a pair of electrons combined, the spin for a cooper pair is an integer and so they can behave like bosons. This effect allows multiple Cooper pairs to occupy the same energy level, unlike single electrons. Superconductors are ideal for qubit systems. They exhibit quantum behaviour on a macroscopic level and the parameters of a superconducting may be modified by changing, for example, the capacitance. There are 3 types of superconducting qubits. The charge qubit, which is based on the number of cooper pairs on either side of a Josephson Junction. The flux qubit, which is based on the quantized magnetic flux inside a superconducting ring. Finally, the phase qubit, which is similar to the charge qubit but instead is based on the amplitude of quantum oscillations on either side of the Josephson Junction. In this report, only the charge qubit is considered.

\subsection{Motivation}
Previous studies on the transmon have mainly focused on the system in either the ground state or first excited levels, both theoretically and experimentally. However, the behaviour of the transmon is not very well described for high states. In experimental work, a transmon is placed in a cavity and the system is driven by an electromagnetic wave. This system shows that for high power readouts, the transmon is ignored and largely has no effect on the overall system. In this investigation, the transmon will be simulated and the effect of a case shall be made for the behaviour of the transmon at high energy states.

\section{Method}
In this section, the theory describing the Josephson Junction and the methods of interpreting the quantum mechanics in such a way that can be replicated using programming techniques.

\subsection{Theory}
As mentioned in the introduction, a Cooper Pair Box is a qubit which utilises the Josephson Junction. It is comprised of a superconducting island, which is connected to a superconducting reservoir via a Josephson Junction. It is then attached to a capacitor and a gate voltage source. The junction has energy $E_J$ and capacitance $C_J$. The box also has an additional energy, the Cooper Pair Coulomb energy:
\begin{equation} \label{eq:1}
E_C =  \frac{(2e)^2}{2C_\Sigma}
\end{equation}
where $C_\Sigma$ is the total capacitance of the box. $E_C$ can also be thought of as the energy required to move one Cooper pair across the junction. We can also consider a charge operator, $\boldsymbol{\hat{n}}$, which is associated with the excess number of cooper pairs on the island. The eigenstates of this operator are such that
\begin{equation} \label{eq:2}
\boldsymbol{\hat{n}}\ket{\boldsymbol{n}} = \boldsymbol{n}\ket{\boldsymbol{n}},  \boldsymbol{n} \in \mathbb{Z}
\end{equation}
The total Hamiltonian of the box consists of two different terms. The first is based on the charge operator and the second relates to the superconducting phase of the island, $\hat{\theta}$. The Hamiltonian for the system is:
\begin{equation} \label{eq:3}
\hatH = E_C (\boldsymbol{\hat{n}} - n_g)^2 - E_J \cos{\hat{\theta}}
\end{equation}
where $n_g$ is the reduced gate charge:
\begin{equation} \label{eq:4}
n_g = \frac{C_g V_g}{2e}
\end{equation}
The charge operator is given in the phase representation as
\begin{equation} \label{eq:5}
\boldsymbol{\hat{n}} = \frac{1}{i} \frac{\delta}{\delta\theta}
\end{equation}
Using this relation, the Hamiltonian can be quantised in the basis of charge eigenstates. Which means that the Hamiltonian can be explicitly expressed  This is the logical basis for the transmon system as $E_C \gg E_J$. The quantised Hamiltonian is:
\begin{equation} \label{eq:6}
\hatH = 4E_C (n-n_g)^2 \ket{n}\bra{n} - \frac{E_J}{2}(\sum_{n} \ket{n+1}\bra{n} + \ket{n}\bra{n+1})
\end{equation}
An alternative method of representing the transmon system is to substitute the Hamiltonian from equation~\ref{eq:3} into the Schr\"odinger equation and solve the equation analytically.
By considering the Hamiltonian in the charge basis it may be solved exactly using Mathieu equations. In this investigation, two different results for the energy of the system have been considered. The first is from Cottet and is given by:
\begin{equation} \label{eq:7}
E_k = E_C \mathcal{M}_A (k+1 - (k+1)[mod(2)] + 2n_g(-1)^k, -\frac{2E_J}{E_C})
\end{equation}
Where $k \in \mathbb{Z}$ and $\mathcal{M}_A$ gives the characteristic value, $a_r$ for even Mathieu functions. However, this only applies to values of $n_g$ such that $0 < n_g < \frac{1}{2}$. The treatment given by Koch expands this such that all values of $n_g$ can be applied. This method also treats $k$ as a function, given by:
\begin{equation} \label{eq:8}
k(m, n_g) = \sum_{l=\pm 1} [int(2n_g + l/2)mod(2)] \times [int(n_g) + l(-1)^m ((m+1)div(2))]
\end{equation}
Where $int$ rounds to the closest integer, $mod$ represents the modulo, and $div$ gives the integer quotient.  The energy of state $m$ is then given by:
\begin{equation} \label{eq:9}
E_m(n_g) = E_C \mathcal{M}_A (2[n_g + k(m,n_g)],-\frac{E_J}{2E_C})
\end{equation}


\section{Results}
\subsection{Initial Results}

\begin{figure}[!ht]
\centering
\includegraphics[width=\linewidth]{C:/Users/Louise/Documents/project/final_project/figs/matrixplotofeigenvectors.png}
\caption{This figure s}
\label{fig:graph1}
\end{figure}
The first step in the investigation is to simulate the transmon in the charge basis, as given by equation~\ref{eq:6}. This was done because it is suspected that the transmon represents a charge state when it is excited to higher energy states. The matrix representation of \ref{eq:6} can be diagonalised. The resulting eigenvectors represent the wavefunctions of the system. Therefore, the eigenvalues give the corresponding energy for each state. The probability function for each state is given by the absolute wavefunction squared. The probability functions are shown in Figure~\ref{fig:graph1}. The intensity of the colour spectrum shows the probability of the system being in that state. The code was written in the Python programming language. As the matrix is tridiagonal, functions from the scipy library for scientific computing were used. Mathematica wasn't considered at this point due to the difficulty in creating tridiagonal matrices in this language.
\begin{figure}[ht]
\centering
\includegraphics[width=\linewidth]{C:/Users/Louise/Documents/project/final_project/figs/plotofeigenvalues.png}
\caption{This figure s}
\label{fig:graph2}
\end{figure}
Following the eigenvector study shown in Figure \ref{fig:graph1}, the eigenvalues were explored to show different results from the same data. This allowed us to see how the energy of the system changes and why this occurs. This is displayed in Figure \ref{fig:graph2}. The y-axis represents the energy of state $m$ proportional to the energy of the junction, $E_J$ to provide a relative scale.

\begin{figure}[ht]
\centering
\includegraphics[width=\linewidth]{C:/Users/Louise/Documents/project/final_project/figs/comparepythonmathematicang0.png}
\caption{writethiis}
\label{fig:graph3}
\end{figure}
Next, the energy of the states was simulated using the analytical approach using Mathieu functions. This was done to ensure the numerical matrix method worked, and also to compare different techniques and sources. First, the treatment by Cottet which is given in equation~\ref{eq:7}. Again the scipy library was used as it contained functions for the characteristic value of the Mathieu function. However, the resulting graph didn't look as expected, which is because it doesn't match the result given by the numerical result in Figure \ref{fig:graph2}. In Figure \ref{fig:graph3} the blue line representing the Python simulation shows anomalies around $25 < m < 50$. In addition, the function used did not accept non-integer parameters as arguments, so some of the accuracies were lost in converting floating point numbers into integers. As a result, the Mathematica technical computing language was used instead for the same method from \ref{eq:7}. As shown in Figure~\ref{fig:graph3}, this gave a result which was much more in keeping with what was expected. showing that the Python result is unphysical and the anomalies are due to the programming used. Going forward, where the methods involving Mathieu equations are written with Mathematica.
Additionally, the range of the energy scale for the states is different in ~\ref{fig:graph3} when compared to the numerical result in ~\ref{fig:graph2}. This could be a simple error of the computation being a factor out of scale, however, it is the same between Mathematica and Python, so the error is likely due to the function used given by equation  ~\ref{eq:7}.
\begin{figure}[ht]
\centering
\includegraphics[width=\linewidth]{C:/Users/Louise/Documents/project/final_project/figs/kochcomparison.png}
\caption{writethiis}
\label{fig:graph4}
\end{figure}
Both due to the errors in the Python implementation and the scaling issues given by the Cottet method, another implementation was considered. The method is given by Koch in equations \ref{eq:8} and \ref{eq:9} was also simulated using Mathematica. Figure \ref{fig:graph4} shows this result, given by the blue curve, compared to the numerical result in \ref{fig:graph2} represented by the orange curve. This shows that the result given by this method is a much better match for the numerical result than the Cottet method, as the curves align almost perfectly, with differences at low energy states.

In the previous graphs, it has been assumed that $n_g = 0$. This is why the graph is composed of 'steps' and isn't smooth at the beginning as with Figure~\ref{fig:graph2}. The difference is seen clearly in Figure  \ref{fig:graph4}. This appears to be a numerical limit that is pertinent to both of the analytical methods.

\begin{figure}[ht]
\centering
\includegraphics[width=\linewidth]{C:/Users/Louise/Documents/project/final_project/figs/coupling.png}
\caption{writethiis}
\label{fig:graph5}
\end{figure}
Next, the coupling between adjacent energy states was investigated. The coupling is effectively how hard it is to between energy states. High coupling means that it is easy to go between states, and low coupling means it is hard. Figure \ref{fig:graph5} shows the coupling between state $m$ and $m+1$, which given by $\bra{m+1}\boldsymbol{\hat{n}}\ket{m}$. This is because the method of moving between states is to add more charge so the charge operator is used. This can be simulated by using matrix multiplication between the eigenvectors produced for Figure \ref{fig:graph1}.
\begin{figure}[ht]
\centering
\includegraphics[width=\linewidth]{C:/Users/Louise/Documents/project/final_project/figs/couplingcomparekoch.png}
\caption{writethiis}
\label{fig:graph6}
\end{figure}
\section{Discussion}
The results of the numerical analysis provide a lot of detail about what happens when the transmon is excited into higher energy states. In low states, where $E_J$ is higher than the energy of the state $E_m$, the system behaves like a harmonic oscillator in a cosine potential, as shown by Koch. This is shown be the region where $m<8$ on Figure \ref{fig:graph1}. As the system approaches the point where $E_J$ and $E_m$ are similar, the wavefunction starts to split into two peaks. This split is an important part of the system. It is shown in Figure~\ref{fig:graph1} where there are two branches or horns that split off, at around $m=10$. This effect is due to the fact that the charge operator can be either positive or negative, as it represents the imbalance of Cooper Pairs on the superconducting island. However, the positive and negative states are degenerate. The degeneracy can be seen clearly in Figure \ref{fig:graph2} and is represented by the 'steps' in the energy. The points where there are two adjacent states with the same energy value are degenerate and effectively indistinguishable from each other. These are shown by the states that make up the horns in Figure \ref{fig:graph1}, such that each degenerate pair is made up of a state from each horn, or a state with a positive charge and another with equal but negative charge. A charge imbalance biased across the junction looks the same as an imbalance of the same amount biased the other way.

This degeneracy is only present at energies such that $E_m > E_J$. In this case, the $E_J$ term in the Hamiltonian has almost no effect on the overall system. Instead, the transmon behaves like a charge state.  As the energy increases, the wavefunction becomes a sharp peak. This is shown in Figure~\ref{fig:graph1}, the intensity of the wavefunction is represented by the colour of the graph. At high values of $m$, the probability of the wavefunction looks almost exactly the same as for the corresponding charge state.

\section{Conclusion}


\end{document}

