%%
%% Author: Louise
%% 08/05/2018
%%

% Preamble
\documentclass[11pt]{article}
\title{High Energy States in Superconducting Cavity Quantum Electrodynamics}
\author{Louise McCann \\ URN: 6307177 \\ Email: lm00293@surrey.ac.uk \\ Department of Physics, University of Surrey, Guildford, GU2 7XH, UK}


% Packages
\usepackage{a4wide}
\usepackage{physics}
\usepackage{graphicx}
\usepackage{amsmath}
\usepackage{amssymb}

\newcommand*{\hatH}{\hat{\mathcal{H}}}


% Document
\begin{document}

    \maketitle

    \begin{abstract}
    Supervisor: Eran Ginossar
    \end{abstract}

    \clearpage

    \section{Introduction}
The superconducting circuit has many uses in modern physics. Its applications include SQUIDS to detect subtle magnetic fields and qubits which are the underlying basis of quantum computers. Quantum computing and the methods behind it could have a big impact on the future of physics. A common circuit made of two superconducting reservoirs with a thin insulating layer between them is called a Josephson Junction. This was predicted by Brian Josephson who won a Nobel Prize for his work. A Josephson Junction works by allowing cooper pairs to tunnel across the barrier without any electrical resistance. Cooper pairs are bound electrons that form at very low temperatures and are the mechanism behind superconductivity. Josephson Junctions are relatively simple devices, which means they are easy to manufacture. They have long coherence times in principle, which means that many operations could be performed on a qubit made out of a Josephson Junction. The Junction is well known and has been studied in many applications. However, the majority of research has been focused on the lower states of the junction. This investigation aims to study the theory of the Josephson Junction and in particular to assess what the properties the system has when excited into higher states.

  \subsection{Quantum Computers}
  Quantum computers are based on the principle that a particle can be in a superposition of states. The theory that computers can be built using this principle has been around since the 1980s and yet the development of quantum computers is still ongoing. These computers would take advantage of quantum algorithms, such as. These could solve problems faster than modern computers and could solve extremely complex problems, such as the many-body problem. Quantum computers are built from qubits, or quantum mechanical circuits that can be measured to be either in the ground state or the first excited state. Qubits rely on the quantum mechanical principle, which states that the qubit is in both states simultaneously until measured. The furthest application has been achieved using NMR techniques to represent a qubit, but another popular medium is to use a superconducting circuit.

 \subsection{Superconducting Circuits}
Superconductors are materials that, at low temperatures, have zero resistance to electronic fields. At these low temperatures, pairs of electrons become bound together in such a way that the energy is lower than that of a single electron. These are the charge carriers in a superconducting material. As a pair of electrons combined, the spin for a cooper pair is an integer and so they can behave like bosons. This effect allows multiple Cooper pairs to occupy the same energy level, unlike single electrons. Superconductors are ideal for qubit systems. They exhibit quantum behavior on a macroscopic level and the parameters of a superconducting may be modified by changing, for example, the capacitance. There are 3 types of superconducting qubits. The charge qubit, which is based on the number of cooper pairs on either side of a Josephson Junction. The flux qubit, which is based on the quantized magnetic flux inside a superconducting ring. Finally, the phase qubit, which is similar to the charge qubit but instead is based on the amplitude of quantum oscillations either side of the Josephson Junction. In this report, only the charge qubit is considered.

  \subsection{Motivation}
Previous studies on the transmon have mainly focused on the system in either the ground state or first excited levels, both theoretically and experimentally. However, the behaviour of the transmon is not very well described for high states. In experimental work, a transmon is placed in a cavity and the system is driven by an electromagnetic wave. This system shows that for high power readouts, the transmon is ignored and largely has no effect on the overall system. In this investigation, the transmon will be simulated and the effect of a case shall be made for the behaviour of the transmon at high energy states.

    \section{Method}
    In this section, the theory describing the Josephson Junction and the methods of interpreting the quantum mechanics in such a way that can be replicated using programming techniques.

   \subsection{Cooper Pair Box}
   As mentioned in the introduction, a Cooper Pair Box is a qubit which utilises the Josephson Junction. It is comprised of a superconducting island, which is connected to a superconducting reservoir via a Josephson Junction. It is then attached to a capacitor and a gate voltage source. The junction has energy $E_J$ and capacitance $C_J$. The box also has an additional energy, the Cooper Pair Coulomb energy:
	\begin{equation}
	E_C =  \frac{(2e)^2}{2C_\Sigma}
	\end{equation}
	where $C_\Sigma$ is the total capacitance of the box. $E_C$ can also be thought of as the energy required to move one cooper pair across the junction. We can also consider a charge operator, $\boldsymbol{\hat{n}}$, which is associated with the excess number of cooper pairs on the island. The eigenstates of this operator are such that
	\begin{equation}
	\boldsymbol{\hat{n}}\ket{\boldsymbol{n}} = \boldsymbol{n}\ket{\boldsymbol{n}},  \boldsymbol{n} \in \mathbb{Z}
	\end{equation}
The total Hamiltonian of the box consists of two different terms. The first is based on the charge operator and the second relates to the superconducting phase of the island, $\hat{\theta}$. The Hamiltonian for the system is:
	\begin{equation} \label{hamil}
	\hatH = E_C (\boldsymbol{\hat{n}} - n_g)^2 - E_J \cos{\hat{\theta}}
	\end{equation}
where $n_g$ is the reduced gate charge:
	\begin{equation}
	n_g = \frac{C_g V_g}{2e}
	\end{equation}
The charge operator is given in the phase representation as
	\begin{equation}
	\boldsymbol{\hat{n}} = \frac{1}{i} \frac{\delta}{\delta\theta}
	\end{equation}
Using this relation, the Hamiltonian can be quantised in the basis of charge eigenstates. This is the logical basis for the transmon system as $E_C \gg E_J$. The quantised Hamiltonian is:
	\begin{equation}
	\hatH = 4E_C (n-n_g)^2 \ket{n}\bra{n} - \frac{E_J}{2}(\sum_{n} \ket{n+1}\bra{n} + \ket{n}\bra{n+1})
	\end{equation}
An alternative method of representing the transmon system is to substitute the Hamiltonian from equation~\ref{hamil} into the Schr\"odinger equation and solve the equation analytically.
    \section{Results}

    \section{Discussion}

    \section{Conclusion}




\end{document}