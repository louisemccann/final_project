%%
%% Author: Louise
%% 08/05/2018
%%

% Preamble
\documentclass[11pt]{article}
\title{High Energy States in Superconducting Cavity Quantum Electrodynamics}
\author{Louise McCann \\ URN: 6307177 \\ Email: lm00293@surrey.ac.uk \\ Department of Physics, University of Surrey, Guildford, GU2 7XH, UK}


% Packages
\usepackage{a4wide}
\usepackage{physics}
\usepackage{graphicx}
\usepackage{amsmath}
\usepackage{amssymb}
\usepackage{caption}
\usepackage{subcaption}

\newcommand*{\hatH}{\hat{\mathcal{H}}}
\graphicspath{C:/Users/Louise/Documents/project/final_project/figs/}

% Document
\begin{document}
\graphicspath{C:/Users/Louise/Documents/project/final_project/figs/}
\maketitle
\begin{center}
Supervisor: Eran Ginossar
\end{center}
\begin{abstract}
Multiple numeric and analytical techniques are used to investigate the high energy states of a Cooper Pair Box based on a Josephson Junction. It is found that for high states the system behaves like a charge state. Also, it is found that degenerate energy states arise due to the fact that charge states can be positive or negative. Several methods are implemented, particularly those from Koch and Cottet. These methods are compared and the validity of the results is discussed.
\end{abstract}

\clearpage

\section{Introduction}
The superconducting circuit has many uses in modern physics. One such use is in Superconducting Quantum Interference Devices or SQUIDs, which are used to detect very small magnetic fields~\cite{jaklevicQuantumInterferenceEffects1964}. Superconducting circuits can also be used to make qubits, which are the underlying mechanism behind quantum computers~\cite{makhlinQuantumStateEngineering2001, youSuperconductingCircuitsQuantum2007}. Quantum computing and the methods behind it could have a big impact on the future of physics, technology and industry~\cite{yingQuantumComputationQuantum2010,montanaroQuantumAlgorithmsOverview2016}.

\subsection{Quantum Computers}
Quantum computers are based on the principle that a particle can be in a superposition of states. The theory that computers can be built using this principle has been around since the 1980's~\cite{feynmanSimulatingPhysicsComputers1982} and yet the development of quantum computers is still ongoing~\cite{harrowQuantumComputationalSupremacy2017}. These computers would take advantage of quantum algorithms. These could solve problems far more quickly than traditional computing and could solve extremely complex problems, such as the many-body problem~\cite{sommaQuantumComputationComplexity2005}. Quantum computers are built from qubits, or quantum mechanical circuits that can be measured to be either in the ground state or the first excited state. Qubits rely on the quantum mechanical principle, which states that the qubit is in a superposition of both states until measured. A full-scale quantum computer has yet to be achieved. The most advanced application has been achieved using Nuclear Magnetic Resonance (NMR) techniques to represent a qubit~\cite{vandersypenExperimentalRealizationShor2001}, but another popular medium is to use a superconducting circuit. Typically, many circuits are linked together to form logic gates, in a similar manner to classical computers.

\subsection{Superconducting Circuits}
Superconductors are materials that, at low temperatures, have zero resistance to electrical fields. At these low temperatures, pairs of electrons become weakly entangled in such a way that the combined energy is lower than that of a single electron. These are the charge carriers in a superconducting material and are known as Cooper pairs~\cite{CooperBoundElectronPairs1956}. The mechanism for superconductivity is typically described by BCS theory~\cite{bardeenTheorySuperconductivity1957}. As a pair of coupled electrons, the spin for a Cooper pair is an integer and so they can behave like bosons. This effect allows multiple Cooper pairs to occupy the same energy level and quantum state, unlike single electrons. Superconductors are ideal for qubit systems. They exhibit quantum behaviour on a macroscopic level and the parameters of a superconducting circuit may be modified by changing, for example, the capacitance of the circuit. There are 3 types of superconducting qubits that can be produced. The charge qubit, which is based on the number of Cooper pairs on either side of a superconducting junction. The flux qubit, which is based on the quantized magnetic flux inside a superconducting ring. Finally, the phase qubit, which is similar to the charge qubit but instead is based on the amplitude of quantum oscillations on either side of the superconducting junction. In this report, only the third of these, the charge qubit is considered.

\subsection{Josephson Junction}
\begin{figure}[ht]
\centering
\includegraphics[width=\linewidth]{C:/Users/Louise/Documents/project/final_project/figs/josephsonjunction.png}
\caption{Schematic of a Josephson Junction and its electrical circuit analogue}
\label{fig:picture1}
\end{figure}
One of the most commonly used superconducting circuits is known as a Josephson Junction, named after Brian Josephson who went on to win a Nobel Prize for his work~\cite{josephsonDiscoveryTunnellingSupercurrents1974,josephsonPossibleNewEffects1962}.
The Josephson Junction is composed of two superconducting reservoirs connected by an insulating barrier.
A Josephson Junction works by allowing Cooper pairs to tunnel across the potential barrier, created by the insulating layer, between the superconducting reservoirs without any electrical resistance when operating at low temperatures.

Josephson Junctions are relatively simple devices, which means they are simple to manufacture. They have long coherence times in principle~\cite{paikObservationHighCoherence2011}, which means that many operations could be performed on a qubit made out of a Josephson Junction. The Junction is well known and has been studied in many applications~\cite{yuCoherentTemporalOscillations2002,simmondsDecoherenceJosephsonPhase2004}. However, the majority of research has been focused on the lower energy states of the junction. This investigation aims to study the theory of the Josephson Junction and in particular to assess what the properties the system has when excited into higher states.

There are a number of ways to introduce a Josephson Junction into a superconducting circuit. A common method is the so-called Cooper Pair Box (CPB)~\cite{bladhSingleCooperpairBox2005}, which is made of a Josephson Junction and a gate capacitor as shown in Figure \ref{fig:picture1}. The CPB will be further explored in the Theory section below. In this report, a form of the CPB known as a transmon is considered. The transmon is a CPB where the energy of the junction is much higher than the energy of the capacitor.

\subsection{Motivation}
Previous studies on the transmon have mainly focused on the system in either the ground state or first excited states, both theoretically and experimentally. However, the behaviour of the transmon is not very well described for higher states above the first excited state. In experimental work, a transmon is placed in a cavity as shown in Figure \ref{fig:picture2} and the system is driven by an electromagnetic standing wave, typically in the microwave region. Recent studies into the best methods of measuring the response of transmon systems have been looking into the effect of high driving power~\cite{reedHighFidelityReadoutCircuit2010}. It is important to understand how the transmon behaves in these high states so that future experiments into superconducting circuits use the transmon to the best functionality. In this investigation, the transmon will be simulated and a case shall be made for the behaviour of the transmon at high energy states.
\begin{figure}[ht]
\centering
\includegraphics[width=\linewidth]{C:/Users/Louise/Documents/project/final_project/figs/cavity.png}
\caption{(a) A schematic of a transmon represented by the dark blue boxes inside a cavity. The waves at the top of the diagram represent the standing wave through the cavity. The other wave beside the transmon represents the resonance frequency of the transmon. (b) Photograph of transmon and cavity system, with detail of transmon.}
\label{fig:picture2}
\end{figure}
\begin{figure}[ht]
\centering
\includegraphics[width=\linewidth]{C:/Users/Louise/Documents/project/final_project/figs/highpowerreadout.png}
\caption{Amplitude of the cavity and transmon response based on measurement power at different frequencies.}
\label{fig:picture3}
\end{figure}
Previous work has looked into how to study and measure a transmon and cavity system, particularly in a high-power readout scheme. Figure~\ref{fig:picture3} shows the response of the transmon system as the driving power is increased with varying frequency. The white arrow shows the expected resonance of the empty cavity without the transmon. However, the measured response frequency is shifted away from this point. As the driving power increases, the system resonance approaches the empty cavity resonance, shown by the black cross. It is suspected that the initial shift is caused by the transmon. The addition of the transmon into the system causes the energy levels of the cavity to shift. This behaviour has been validated in other research~\cite{bishopResponseStronglyDriven2010,paikObservationHighCoherence2011,elliottApplicationsFokkerPlanckEquation2016}. However, at higher power this effect disappears and the transmon is effectively ignored by the system. This report aims to explore why this occurs.


\section{Theory}
In this section, the theory describing the Josephson Junction and the methods of interpreting the quantum mechanics are presented in such a way that can be replicated using programming techniques.

As mentioned in the introduction, a Cooper Pair Box is a qubit which is made of a Josephson Junction. It is comprised of a superconducting island, which is connected to a superconducting reservoir via a Josephson Junction. It is then attached to a capacitor and a gate voltage source~\cite{cottetImplementationQuantumBit2002}.
\begin{figure}[ht]
\centering
\includegraphics[width=\linewidth]{C:/Users/Louise/Documents/project/final_project/figs/cpb.png}
\caption{Schematic of a Copper Pair Box (CPB) and its electrical circuit analogue}
\label{fig:picture4}
\end{figure}
The junction has energy $E_J$ and capacitance $C_J$. The CPB also has an additional energy, the Cooper Pair Coulomb energy:
\begin{equation} \label{eq:1}
E_C = \frac{(2e)^2}{2C_\Sigma}
\end{equation}
Where $C_\Sigma$ is the total capacitance of the CPB. $E_C$ can also be thought of as the energy required to move one Cooper pair across the junction. We can also consider a charge operator, $\boldsymbol{\hat{n}}$, which is associated with the excess number of Cooper pairs on the island. The eigenstates of the charge operator are such that
\begin{equation} \label{eq:2}
\boldsymbol{\hat{n}}\ket{\boldsymbol{n}} = \boldsymbol{n}\ket{\boldsymbol{n}}, \boldsymbol{n} \in \mathbb{Z}
\end{equation}
The total Hamiltonian of the box consists of two different terms as shown in Equation~\ref{eq:3}. The first term is based on the charge operator and the second relates to the superconducting phase of the island, $\hat{\theta}$. The Hamiltonian for the system is:
\begin{equation} \label{eq:3}
\hatH = E_C (\boldsymbol{\hat{n}} - n_g)^2 - E_J \cos{\hat{\theta}}
\end{equation}
where $n_g$ is the reduced gate charge:
\begin{equation} \label{eq:4}
n_g = \frac{C_g V_g}{2e}
\end{equation}
The charge operator is given in the phase representation as
\begin{equation} \label{eq:5}
\boldsymbol{\hat{n}} = \frac{1}{i} \frac{\delta}{\delta\theta}
\end{equation}
Using this relation, the Hamiltonian can be quantised in the basis of charge eigenstates. Which means that the Hamiltonian can be explicitly expressed in terms of charge states.This is the logical basis for the transmon system as $E_C \gg E_J$. The quantised Hamiltonian is:
\begin{equation} \label{eq:6}
\hatH = 4E_C (n-n_g)^2 \ket{n}\bra{n} - \frac{E_J}{2}(\sum_{n} \ket{n+1}\bra{n} + \ket{n}\bra{n+1})
\end{equation}
An alternative method of representing the transmon system is to substitute the Hamiltonian from Equation~\ref{eq:3} into the Schr\"odinger equation and solve the equation analytically.
By considering the Hamiltonian in the charge basis it may be solved exactly using Mathieu equations~\cite{rubyApplicationsMathieuEquation1996}. In this investigation, two different models for the energy of the system have been considered. The first is from Cottet~\cite{cottetImplementationQuantumBit2002} and is given by:
\begin{equation} \label{eq:7}
E_k = E_C \mathcal{M}_A (k+1 - (k+1)[mod(2)] + 2n_g(-1)^k, -\frac{2E_J}{E_C})
\end{equation}
Where $k \in \mathbb{Z}$ gives the state of the system and $\mathcal{M}_A$ gives the characteristic value for even Mathieu functions. However, this only applies to values of $n_g$ such that $0 < n_g < \frac{1}{2}$. The treatment given as an alternate model by Koch~\cite{kochChargeinsensitiveQubitDesign2007} expands this such that all values of $n_g$ can be applied. This method also treats $k$ as a function, given by:
\begin{equation} \label{eq:8}
k(m, n_g) = \sum_{l=\pm 1} [int(2n_g + l/2)mod(2)] \times [int(n_g) + l(-1)^m ((m+1)div(2))]
\end{equation}
Where $int$ rounds to the closest integer, $mod$ represents the modulo, and $div$ gives the integer quotient. The energy of state $m$ is then given by:
\begin{equation} \label{eq:9}
E_m(n_g) = E_C \mathcal{M}_A (2[n_g + k(m,n_g)],-\frac{E_J}{2E_C})
\end{equation}

The equations given form a complete model of the transmon. They can be easily converted into functions using appropriate software and then manipulated and studied computationally.

\section{Results}

\begin{figure}[ht]
\centering
\includegraphics[width=\linewidth]{C:/Users/Louise/Documents/project/final_project/figs/matrixplotofeigenvectors.png}
\caption{Probability of the system being in a charge state $n$ based on the Energy State $m$ }
\label{fig:graph1}
\end{figure}
The first step in the investigation is to simulate the transmon in the charge basis, as given by Equation~\ref{eq:6}. This was done because it is suspected that the transmon represents a charge state when it is excited to higher energy states. The matrix representation of Equation~\ref{eq:6} can be diagonalized. The resulting eigenvectors represent the wavefunctions of the system. Hence, the eigenvalues give the corresponding energy for each state. The probability function for each state is given by the absolute wavefunction squared. The probability functions are shown in Figure~\ref{fig:graph1}. The intensity of the colour spectrum shows the probability of the system being in that state. The code was written in the Python programming language. As the matrix is tridiagonal, functions from the SciPy library for scientific computing were used~\cite{jonesSciPyOpenSource2001}. The technical computing language Mathematica wasn't considered at this point due to the difficulty in creating tridiagonal matrices in this language.
\begin{figure}[ht]
\centering
\includegraphics[width=\linewidth]{C:/Users/Louise/Documents/project/final_project/figs/plotofeigenvalues.png}
\caption{This figure s}
\label{fig:graph2}
\end{figure}
Following the eigenvector results shown in Figure~\ref{fig:graph1}, the eigenvalues were explored to show different results from the same data, which allows the energy of the system to be explored. This is displayed in Figure~\ref{fig:graph2}. The y-axis represents the energy of state $m$ proportional to the energy of the junction, $E_J$ to provide a relative scale.

\begin{figure}[ht]
\centering
\includegraphics[width=\linewidth]{C:/Users/Louise/Documents/project/final_project/figs/comparepythonmathematica.png}
\caption{writethiis}
\label{fig:graph3}
\end{figure}
Next, the energy of the states was simulated with an analytical approach using Mathieu functions. This was done to ensure the numerical matrix method worked, and also to compare different modelling techniques and sources. First, the treatment by Cottet which is given in Equation~\ref{eq:7}. Again, the SciPy library was used as it contained functions for the characteristic value of the Mathieu function. However, the resulting graph did not look as expected, which is because it doesn't match the result given by the numerical result in Figure~\ref{fig:graph2}. In Figure~\ref{fig:graph3} the blue line representing the Python simulation shows anomalies around $25 < m < 50$. In addition, the function used did not accept non-integer parameters as arguments, so some of the accuracies were lost in converting floating point numbers into integers. As an alternative, the Mathematica technical computing language was used to verify the results from the Python simulation using the same method from Equation~\ref{eq:7}. As shown in Figure~\ref{fig:graph3}, this gave a result which was much more in keeping with what was expected, Showing that the Python result is unphysical and the anomalies are due to the programming language used. Going forward, the methods involving Mathieu equations are written with Mathematica.
An additional difference between numerical and analytical results is that the range of the energy scale for the states is different in Figure ~\ref{fig:graph3} when compared to the numerical result in Figure ~\ref{fig:graph2}. This could be a simple error of the computation being a factor out of scale, however, the same discrepancy is the same between Mathematica and Python, so the error is likely due to the function used, given by equation ~\ref{eq:7}.
\begin{figure}[ht]
\centering
\includegraphics[width=\linewidth]{C:/Users/Louise/Documents/project/final_project/figs/kochcomparison.png}
\caption{writethiis}
\label{fig:graph4}
\end{figure}

Both due to the errors in the Python implementation and the scaling issues given by the Cottet method, another implementation was considered. The method given by Koch in equations~\ref{eq:8} and~\ref{eq:9} was also simulated using Mathematica. Figure~\ref{fig:graph4} shows this result, given by the blue curve, compared to the numerical result in Figure~\ref{fig:graph2} represented by the orange curve. This shows that the result given by this method is a much better match for the numerical result than the Cottet method, as the curves agree almost perfectly, with differences at low energy states.

In the previous graphs, it has been assumed that $n_g = 0$. This has been used to simplify the results and reduce the number of initial variables. This is why the graph is composed of 'steps' and isn't smooth at low energies as with Figure~\ref{fig:graph2}. The difference is seen clearly in Figure~\ref{fig:graph4}. This appears to be a numerical limit that is pertinent to both of the analytical methods.

\begin{figure}[ht]
\centering
\includegraphics[width=\linewidth]{C:/Users/Louise/Documents/project/final_project/figs/coupling.png}
\caption{writethiis}
\label{fig:graph5}
\end{figure}

Next, the coupling between adjacent energy states was investigated. The coupling is effectively how hard it is to transition between energy states. The high coupling means that it is easy to go between states, and low coupling means it is difficult. Figure~\ref{fig:graph5} shows the coupling between state $m$ and $m+1$, which given by $\bra{m+1}\boldsymbol{\hat{n}}\ket{m}$. This is because the method of moving between states is to add more charge, so the charge operator is used. This can be simulated by using matrix multiplication between the eigenvectors calculated for Figure~\ref{fig:graph1}.

\section{Discussion}
\subsection{Main Discussion}
The results of the numerical analysis provide a lot of detail about what happens when the transmon is excited into higher energy states. In low states, where $E_J$ is higher than the energy of the state $E_m$, the system behaves like a harmonic oscillator in a cosine potential, as shown by Koch. This is shown to be the region where $m<8$ on Figure~\ref{fig:graph1}. As the system approaches the point where $E_J$ and $E_m$ are similar, the wavefunction starts to split into two peaks. This split is an important part of the system. It can be seen in Figure~\ref{fig:graph1}, where there are two branches or horns that split off, at around $m=10$. This effect is due to the fact that the charge operator can be either positive or negative, as it represents the imbalance of Cooper Pairs on the superconducting island. However, the positive and negative states are degenerate. The degeneracy can be seen clearly in Figure~\ref{fig:graph2} and is represented by the 'steps' in the energy. The points where there are two adjacent states with the same energy value are degenerate and effectively indistinguishable from each other. These are shown by the states that make up the horns in Figure~\ref{fig:graph1}, such that each degenerate pair is made up of a state from each horn, or a state with a positive charge and another with equal but negative charge. A charge imbalance biased across the junction looks the same as an imbalance of the same amount biased the other way.

This degeneracy is only present at energies such that $E_m > E_J$. This is best shown in Figure~\ref{fig:graph2} where the smooth section of the curve rapidly changes into steps where $E_m = E_J$. The smoothness before this is largely due to the $E_J$ term given in the Hamiltonian, Equation~\ref{eq:3}. After this point, the $E_J$ term in the Hamiltonian has almost no effect on the overall system. Instead, the transmon behaves like a charge state and entirely dependent on the charge operator. This is likely because there is such a large charge imbalance that the charged term dominates. A charge state is represented as a wavefunction by an intense peak at the corresponding charge, as it highly probable that the state will be at that charge value. As the energy increases and the transmon is forced into higher states, the peak of the wavefunction becomes sharper. This is shown in Figure~\ref{fig:graph1}, the intensity of the wavefunction is represented by the colour of the graph. At high values of $m$, the probability of the wavefunction looks almost exactly the same as for the corresponding charge state.

\subsection{Second Discussion}
The analytical methods presented by Cottet and Koch both give appreciable results when compared to the numerical function, as shown in Figure~\ref{fig:graph4}. They both follow the same overall trend and the important steps produced by the degenerate states are present. However, for the case where $n_g = 0$, they do not accurately represent the lower states. Neither method achieves the smooth section of the curve for the region before $E_m = E_J$. This can be rectified by making $n_g$ very small, for example setting $n_g = 10^{-10}$. This is shown in Figure~\ref{fig:graph6}, where there is a smooth section. However, the due to scaling issues the numerical method remains the better alternative. This emphasises how effective and comparatively simple the numerical solution is.

\begin{figure}[ht]
\centering
\includegraphics[width=\linewidth]{C:/Users/Louise/Documents/project/final_project/figs/compare_analytical_numerical.png}
\caption{writethiis}
\label{fig:graph6}
\end{figure}



\subsection{Coupling Constant}

The wavefunctions for the states produced by the numerical results can be used to investigate how the states are coupled to each other. This is displayed in Figure~\ref{fig:graph5}, which shows the coupling between state $m$ and the adjacent state $m+1$. This graph can be split into 3 effectively different regions.

The first region is for $m<10$. For this region the coupling between adjacent states increases, so the system can relatively easily transition between states. This is approximately the extent to which the transmon has been studied so far. An approximation is given by Koch as:
\begin{equation} \label{eq:10}
\bra{m+1}\boldsymbol{\hat{n}}\ket{m} \approx \sqrt{\frac{m+1}{2}} \bigg( \frac{E_J}{8E_C}\bigg)^{1/4}
\end{equation}
\begin{figure}[ht]
\centering
\includegraphics[width=\linewidth]{C:/Users/Louise/Documents/project/final_project/figs/couplingcomparekoch.png}
\caption{writethiis}
\label{fig:graph7}
\end{figure}
This function is shown alongside the first few states of the numerical coupling constant is displayed in Figure~\ref{fig:graph7}. This graph shows that for low states, the Koch approximation is a good representation for the coupling constant between adjacent energy states. However, the numerical result and the approximation diverge above $m=7$, so this means that the approximation is only relevant for these very low energy states and should not be extended.

The second region of Figure~\ref{fig:graph5} to be discussed is for $10<m<15$. During this section, the curve decreases smoothly at a constant rate. This matches up to the region in Figure~\ref{fig:graph1} where the graph has split into two horns but the states are not completely charge-like, which occurs around $m=15$, where $E_m = E_J$. This is likely because of the nature of the potential. It is shaped like a cosine, which means that at the very top of the potential, the effective barrier between potential wells is thin, and it is much easier for the wavefunction to tunnel through the wall. Hence at this point, the transmon partially behaves like a charge state because it is able to partially tunnel through the potential. When the energy $E_m$ is much higher than $E_J$, there is no longer a potential well and the transmon behaves almost completely like a charge state. The coupling between the states decreases rapidly to the point where it approaches 0.

The final region of Figure~\ref{fig:graph5}, where $m>30$, is interesting as it is composed of jagged spikes, while the average trend remains flat and approaches 0. The points are not periodic so this region would be ideal to investigate further in order to confirm the result and ensure that it is not due to computational errors. However, a possible reason for the jagged nature of the curve is the degenerate energy states. As discussed previously, the steps on Figure~\ref{fig:graph2} are caused by degenerate states $\ket{-n}$ and $\ket{n}$. In this case, the graph of the energy goes up in states $\ket{n}$, $\ket{-n}$ and then $\ket{n+1}$ ,$\ket{-(n+1)}$, or something to that effect. This switching between opposite states and adjacent states may be the reason behind the jagged spikes. The tips of the spikes correspond to the relatively high coupling between the positive and negative states, while the minima are attributed to adjacent states. Coupling between adjacent states is expected to be lowered as energy increases, which also explains the overall trend towards 0. This is due to the Coulomb effect. In this region of high energy states, the transom behaves like a charge state. This is because of an excess of Cooper pairs on one side of the junction. Cooper pairs are composed of coupled electrons, which have a negative charge. This means that the total charge of the Cooper pair is also negative. In order to progress to the next $\ket{n+1}$ state, another Cooper pair must tunnel across the junction and increase the excess. However, due to the amount of negative charge already accumulated, the Coulomb potential will repel the negative charge and energy will be required to successfully move it. As more charge accumulates, it becomes more difficult to overcome the Coulomb potential and it will require more energy to push the system into the $\ket{n+1}$ state. This is why the states have a low coupling because it is less unlikely that the system will have enough energy to potentially move up adjacent states.

The fact that the system behaves like a charge state goes some way to explaining the shift in resonance found in experimental research. The low coupling suggests that at this point a large amount of energy is required to move to adjacent states. If the driving power is not sufficient enough to affect the transmon, there will be no coupling between the transmon and the cavity. The energy levels of the cavity will behave as they would if they were empty, and the presence of the transmon will not shift the resonance of the system. As the energy increases, the transmon is effectively ignored by the cavity as it is no longer affected by the driving power.


\section{Conclusion}
The high energy states of the transmon have been investigated computationally in order to help illuminate new experimental work on this system~\cite{lescanneDynamicsOffresonantlyPumped2018}. Results show that the high energy states strongly resemble charge states with degeneracies arising for the energies states $\ket{n}$ and $\ket{-n}$. The low coupling of adjacent states, which represents the difficulty of moving between the states, was found to reinforce the theory that the system represents a charge state. Both a numerical approach, based upon looking at the Hamiltonian of the system in the charge basis, and analytical approaches, based upon the work by Cottet~\cite{cottetImplementationQuantumBit2002} and Koch~\cite{kochChargeinsensitiveQubitDesign2007} were used. When implementing the analytical techniques, it was found that there were errors arising from the particular programming language used, emphasising the importance of a careful choice of programming language. Comparison of the numerical and analytical techniques found that there were broadly in agreement. However, the numerical approach was preferred due to the better treatment over a wide range of energy states, while the analytic approaches were approximations which worked primarily on a specific range of parameters including the gate charge, $n_g$. These results, if confirmed by experiment, will allow a good understanding of the transmon using a charge state model.

\clearpage
\bibliography{references}
\bibliographystyle{ieeetr}
\appendix

\end{document}



